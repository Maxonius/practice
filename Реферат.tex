\abstract{РЕФЕРАТ}

Объем работы равен \formbytotal{lastpage}{страниц}{е}{ам}{ам}. Работа содержит \formbytotal{figurecnt}{иллюстраци}{ю}{и}{й}, \formbytotal{tablecnt}{таблиц}{у}{ы}{}, \arabic{bibcount} библиографических источников и \formbytotal{числоПлакатов}{лист}{}{а}{ов} графического материала. Количество приложений – 2. Графический материал представлен в приложении А. Фрагменты исходного кода представлены в приложении Б.

Перечень ключевых слов: цифровой туризм, веб-приложение, карта, маршрут, туристические объекты, API Яндекс.Карт, JavaScript, PHP, фильтрация, маршрутизация, геолокация, база данных, пользователь, интерфейс, взаимодействие, координаты, балун, точка интереса, отзыв, избранное, навигация, интерактивность, адаптивность, интерактивная карта, информационная система, региональный туризм, разработка, проектирование.

Объектом разработки является веб-приложение, предназначенное для поддержки туристической деятельности в Курской области с использованием интерактивной карты и пользовательских маршрутов.

Целью выпускной квалификационной работы является повышение доступности информации о достопримечательностях региона, улучшение пользовательского взаимодействия с туристическими данными, а также предоставление возможности планирования маршрутов с помощью цифровых технологий.

В ходе разработки были определены ключевые сценарии взаимодействия с системой, реализована архитектура клиентской и серверной частей, организовано хранение и обработка географических данных, обеспечено взаимодействие с картографическим API Яндекс.Карт. Разработан функционал отображения точек интереса, фильтрации по категориям, построения линейных и кольцевых маршрутов, добавления в избранное и управления пользовательским контентом (описания, изображения, отзывы).

При реализации проекта использовались языки JavaScript и PHP, технологии асинхронного взаимодействия с сервером, а также REST-интерфейс для обмена данными. Пользовательский интерфейс адаптирован под разные устройства, поддерживает динамическую подгрузку контента и нацелен на максимальное удобство взаимодействия.

Разработанное приложение протестировано, собрано в единую систему и может быть использовано как самостоятельный туристический сервис либо интегрировано в инфраструктуру цифрового туризма региона.

\selectlanguage{english}
\abstract{ABSTRACT}
  
The volume of work is \formbytotal{lastpage}{page}{}{s}{s}. The work contains \formbytotal{figurecnt}{illustration}{}{s}{s}, \formbytotal{tablecnt}{table}{}{s}{s}, \arabic{bibcount} bibliographic sources and \formbytotal{числоПлакатов}{sheet}{}{s}{s} of graphic material. The number of applications is 2. The graphic material is presented in annex A. The layout of the site, including the connection of components, is presented in annex B.

List of keywords: digital tourism, web application, map, route, tourist attractions, Yandex.Maps API, JavaScript, PHP, filtering, routing, geolocation, database, user, interface, interaction, coordinates, balloon, point of interest, review, favorites, navigation, interactivity, adaptability, interactive map, information system, regional tourism, development, design.

The object of development is a web application designed to support tourism activities in the Kursk region through the use of an interactive map and user-defined routes.

The aim of the final qualifying project is to increase accessibility of information about local landmarks, improve user interaction with tourist data, and provide route planning capabilities using digital technologies.

During the development process, the main user interaction scenarios were defined, the architecture of both client and server parts was implemented, storage and processing of geographic data were organized, and integration with the Yandex.Maps API was carried out. Functionality was developed for displaying points of interest, filtering by category, building linear and circular routes, adding favorites, and managing user content (descriptions, images, reviews).

The project was implemented using JavaScript and PHP, asynchronous communication technologies, and REST interfaces for data exchange. The user interface is adapted for various devices, supports dynamic content loading, and is focused on providing a seamless user experience.

The developed application has been tested, assembled into a complete system, and can be used as a standalone tourist service or integrated into a regional digital tourism infrastructure.
\selectlanguage{russian}
