\abstract{РЕФЕРАТ}

Объем работы равен \formbytotal{lastpage}{страниц}{е}{ам}{ам}. Работа содержит \formbytotal{figurecnt}{иллюстраци}{ю}{и}{й}, \formbytotal{tablecnt}{таблиц}{у}{ы}{}, \arabic{bibcount} библиографических источников и \formbytotal{числоПлакатов}{лист}{}{а}{ов} графического материала. Количество приложений – 2. Графический материал представлен в приложении А. Фрагменты исходного кода представлены в приложении Б.

Перечень ключевых слов: цифровой туризм, веб-приложение, карта, маршрут, туристические объекты, API Яндекс.Карт, JavaScript, PHP, фильтрация, маршрутизация, геолокация, база данных, пользователь, интерфейс, координаты, балун, отзыв, избранное, навигация, интерактивная карта, информационная система, региональный туризм.

Объектом разработки является веб-приложение, предназначенное для поддержки туристической деятельности в Курской области с использованием интерактивной карты и пользовательских маршрутов.

Целью выпускной квалификационной работы является повышение доступности информации о достопримечательностях региона, а также предоставление возможности планирования маршрутов с помощью цифровых технологий.

В ходе разработки были определены ключевые сценарии взаимодействия с системой, реализована архитектура клиентской и серверной частей, организовано хранение и обработка географических данных, обеспечено взаимодействие с картографическим API Яндекс.Карт.

При реализации проекта использовались языки JavaScript и PHP, технологии асинхронного взаимодействия с сервером, а также REST-интерфейс для обмена данными. Пользовательский интерфейс адаптирован под разные устройства, поддерживает динамическую подгрузку контента и нацелен на максимальное удобство взаимодействия.

Разработанное приложение протестировано, собрано в единую систему и может быть использовано как самостоятельный туристический сервис либо интегрировано в инфраструктуру цифрового туризма региона.

\selectlanguage{english}
\abstract{ABSTRACT}
  
The volume of work is \formbytotal{lastpage}{page}{}{s}{s}. The work contains \formbytotal{figurecnt}{illustration}{}{s}{s}, \formbytotal{tablecnt}{table}{}{s}{s}, \arabic{bibcount} bibliographic sources and \formbytotal{числоПлакатов}{sheet}{}{s}{s} of graphic material. The number of applications is 2. The graphic material is presented in annex A. The layout of the site, including the connection of components, is presented in annex B.

List of keywords: digital tourism, web application, map, route, tourist attractions, Yandex.Maps API, JavaScript, PHP, filtering, routing, geolocation, database, user, interface, coordinates, balloon, review, favorites, navigation, interactive map, information system, regional tourism.

The object of this project is a web application designed to support tourist activity in the Kursk region through the use of an interactive map and custom route planning.

The purpose of this final qualification work is to increase the accessibility of information about regional attractions and to provide users with the ability to plan routes using digital technologies.

During the development process, key user interaction scenarios were defined, the architecture of both client-side and server-side components was implemented, geographic data storage and processing were organized, and integration with the Yandex.Maps API was ensured.

The project was developed using JavaScript and PHP, along with asynchronous server communication technologies and a REST interface for data exchange. The user interface is responsive, supports dynamic content loading, and is focused on maximizing usability across devices.

The resulting application has been tested, assembled into a unified system, and can be used either as an independent tourist service or as part of the regional digital tourism infrastructure.
\selectlanguage{russian}
