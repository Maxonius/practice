\section*{ОБОЗНАЧЕНИЯ И СОКРАЩЕНИЯ}

API (Application Programming Interface) -- программный интерфейс взаимодействия между компонентами системы или внешними сервисами. В контексте данной работы используется API Яндекс.Карт для отображения объектов, построения маршрутов и выполнения геокодирования.

HTTP (HyperText Transfer Protocol) -- протокол передачи гипертекста, используемый для обмена данными между клиентом и сервером в сети интернет.

HTTPS (HTTP Secure) -- защищённая версия протокола HTTP, использующая SSL или TLS для шифрования передаваемых данных.

Nginx -- высокопроизводительный веб-сервер и обратный прокси-сервер, применяемый для обработки HTTP(S)-запросов, распределения нагрузки и маршрутизации запросов к backend-серверу.

Backend -- серверная часть веб-приложения, отвечающая за обработку логики, работу с данными, хранение информации и выполнение действий по запросу клиента. В рамках данной работы backend реализован на PHP и взаимодействует с хранилищем JSON-файлов.

Балун -- всплывающее окно на карте, отображающее информацию об объекте (название, описание, изображение, кнопки действия).

Геолокация -- определение текущего местоположения пользователя с помощью встроенных функций браузера и устройства.

UML (Unified Modelling Language) -- язык графического описания для объектного моделирования в области разработки программного обеспечения.
