\section{Анализ предметной области}
\subsection{Актуальность цифровизации в сфере туризма}

Современное развитие туристической отрасли невозможно без активного внедрения и использования цифровых технологий. Сегодня путешествие начинается задолго до фактического выезда — с онлайн-поиска информации, просмотра интерактивных карт, планирования маршрутов и бронирования услуг. В условиях стремительного роста числа пользователей интернета и мобильных устройств изменились как поведение туристов, так и ожидания к информационному сопровождению поездки.

Цифровизация позволяет решить ряд ключевых проблем, характерных для традиционного туризма: недостаток актуальной информации о локациях, сложность построения маршрутов с учётом интересов конкретного путешественника, отсутствие персонализации и слабая вовлечённость в культурный контекст региона. Веб-приложения с картами достопримечательностей, возможностью планирования и сохранения маршрутов, фильтрации объектов по категориям становятся эффективным инструментом для популяризации как крупных, так и малоизвестных туристических направлений.

Особенно значима цифровизация в контексте внутреннего туризма. Во многих регионах России, включая Курскую область, есть богатый культурный, исторический и природный потенциал, который остаётся малозаметным для широкой аудитории именно из-за недостатка информационной доступности. Создание цифровых сервисов помогает изменить это положение, открывая туристам новые маршруты и объекты, в том числе вне столиц и крупных городов.

Тема цифровизации туризма развивается также благодаря общими глобальными тенденциями. Согласно исследованиям Всемирной туристической организации (UNWTO)\cite{b1}, более 70\% путешественников перед поездкой используют онлайн-карты и мобильные приложения для поиска информации о местах. Цифровые технологии в туризме уже перестали быть инновацией — это ожидаемый стандарт взаимодействия с клиентом, в том числе и на уровне муниципалитетов и регионов.

Также стоит отметить растущую популярность самостоятельных путешествий, где пользователь сам выбирает интересные объекты, формирует уникальный маршрут и предпочитает гибкость вместо готовых туров. Такой подход требует инструментов, позволяющих быстро сориентироваться на местности, найти интересные точки и спланировать передвижение. В этом контексте веб-приложения с интерактивной картой становятся не просто дополнением, а основой цифровой инфраструктуры туристического продукта.

Цифровизация повышает доступность информации, способствует развитию локального туризма, поддерживает устойчивый подход к использованию природных и культурных ресурсов региона\cite{b2}. Кроме того, цифровые платформы позволяют формировать обратную связь, собирать пользовательскую аналитику и на её основе улучшать туристический опыт.

Таким образом, внедрение цифровых технологий в сферу туризма представляет собой не временную тенденцию, а необходимость, обусловленную актуальными социально-экономическими и технологическими изменениями. Разработка веб-приложения с картой туристических объектов региона рассматривается как практико-ориентированный вклад в формирование современной цифровой инфраструктуры, направленной на повышение привлекательности локальных маршрутов, улучшение доступности и полноты информационного сопровождения, а также на стимулирование устойчивого и контролируемого туристического потока.

\subsection{Туристический потенциал}

Курская область обладает большим туристическим потенциалом, основанным на сочетании историко-культурного наследия, памятных мест, природных достопримечательностей и различных мероприятий. Несмотря на относительную скромность по количеству туристов по сравнению с крупнейшими регионами России, область сохраняет устойчивый интерес среди путешественников, стремящихся к осмысленному и содержательному отдыху.

Культурно-исторический туризм.
Регион богат архитектурными памятниками, связанными с духовной и светской историей. Среди наиболее значимых объектов:
\begin{itemize}
	\item Коренная пустынь — один из древнейших православных монастырей, основанный в XIII веке, ежегодно притягивает тысячи паломников и туристов;
	\item Знаменский собор в Курске — выдающийся образец церковной архитектуры XVIII века;
	\item Храм Александра Невского — православный храм в городе Обояни Курской области. Храм был заложен 19 мая 1891 года на Александро-Невской площади города Обояни епископом Курским и Белгородским Иустином.
\end{itemize}

В Курской области работает ряд музеев, наиболее значимые из которых — Краеведческий музей, Музей Курской битвы, Дом-музей Георгия Свиридова. Экспозиции раскрывают богатую историю региона от древности до современности.

Военно-патриотический туризм.
Курская область имеет исключительное значение как территория проведения Курской битвы — ключевого сражения Второй мировой войны. Памятники, мемориальные комплексы и музеи, посвящённые этим событиям, формируют основу патриотических маршрутов. Среди них — Мемориальный комплекс «Курская битва», Мемориальный комплекс «Курская дуга», Мемориальный комплекс «Памяти павших в годы Великой Отечественной войны».

Природный и экологический туризм.
Богатая природа региона позволяет развивать экотуризм. Среди ключевых направлений:
\begin{itemize}
	\item Центрально-Черноземный государственный природный биосферный заповедник имени профессора В. В. Алехина — объект под охраной ЮНЕСКО, с уникальной флорой и фауной;
	\item живописные берега рек Сейм и Псел, привлекательные для любителей активного отдыха — сплавов, рыбалки, прогулок;
	\item лесопарковые зоны и курганы, подходящие для однодневных маршрутов и наблюдений за природой.
\end{itemize}

Событийный туризм.
Курская область активно развивает событийные направления. Регулярно проходят:
\begin{itemize}
	\item Курская Коренская ярмарка — возрождённая традиция крупной торгово-культурной выставки;
	\item музыкальные фестивали в честь композитора Георгия Свиридова;
	\item региональные исторические реконструкции и православные праздники.
\end{itemize}

Паломнический туризм.
Наличие значимых религиозных объектов и действующих монастырей создаёт предпосылки для развития паломнических маршрутов. Кроме Коренной пустыни, интерес представляют Свято-Троицкий монастырь, Успенская церковь в Курчатове и другие храмы, многие из которых включены в организованные туры.

Потенциал развития.
Несмотря на богатый ресурсный фундамент, туристическая сфера региона остаётся неизвестной и малозаметной в цифровом пространстве. Это открывает возможности для внедрения веб-сервисов, картографических решений и цифровых платформ, способных повысить узнаваемость объектов, упростить навигацию и привлечь новых туристов как из России, так и из-за рубежа.

\subsection{Основные задачи систем цифрового туризма}

Системы цифрового туризма представляют собой совокупность информационных, технических и программных решений, обеспечивающих эффективное взаимодействие между туристом, территорией и цифровой инфраструктурой. Их главная цель — упростить и улучшить пользовательский опыт на всех этапах путешествия: от планирования до самостоятельного исследования местности. Такие системы не ограничиваются банальным отображением достопримечательностей на карте — они выстраивают полноценную информационную экосистему, опирающуюся на актуальные данные, навигационные функции, персонализацию и вовлечённость пользователей.

Одной из ключевых задач цифровых платформ в туризме является сбор и структурированное представление информации о туристических объектах. Это включает не только географическое расположение, но и подробное описание, фотографии, исторические справки, режимы работы, контактные данные и другие параметры, необходимые для осознанного выбора путешественником. Благодаря цифровым решениям данные о местах интереса становятся доступными в любое время и на любом устройстве.

Следующая важная задача — визуализация объектов с помощью интерактивных карт. Пользователь получает возможность ориентироваться в незнакомом пространстве, оценивать расстояния, просматривать распределение объектов по территории и выбирать оптимальные направления. Интеграция с картографическими API позволяет дополнить данные визуальными и функциональными средствами: метками, балунами, фильтрами, маршрутизацией и другими средствами взаимодействия с пространством.

Третья задача цифровых туристических систем — обеспечение пользовательского участия и персонализации контента. Современные приложения всё чаще предлагают возможность сохранять избранные места, формировать собственные маршруты, оставлять отзывы, оценивать достопримечательности и делиться впечатлениями. Это не только увеличивает вовлечённость аудитории, но и позволяет системе «учиться» и адаптироваться под интересы разных типов пользователей — от любителей активного отдыха до ценителей культурного туризма.

Также важной задачей выступает механизм построения маршрутов с учётом пользовательских предпочтений. В отличие от стандартных навигационных систем, туристические платформы должны уметь работать с произвольными точками, промежуточными остановками, временем посещения, типом маршрута (пеший, велосипедный, автомобильный) и даже событиями или погодными условиями. Гибкость маршрутного функционала — это конкурентное преимущество таких приложений.

Немаловажной задачей является доступ к информации в оффлайн-режиме. Не все туристические направления обеспечены стабильным интернет-соединением. Поэтому возможность заранее сохранить данные о маршрутах, объектах и инструкциях — критически важна для обеспечения автономности путешественника.

Системы цифрового туризма также выполняют рекламно-просветительскую функцию. Они способствуют продвижению культурных, природных и исторических объектов, повышая их узнаваемость и включённость в региональные и национальные маршруты. Для территорий с низким туристическим трафиком это может стать инструментом привлечения внимания и развития локальной экономики.

Важным направлением работы таких систем становится анализ поведения пользователей и статистика посещаемости объектов, которая может быть полезна органам власти, учреждениям культуры, администрациям парков и другим заинтересованным сторонам. Аналитика помогает принимать обоснованные управленческие решения, выявлять потребности туристов и своевременно реагировать на изменения интересов.

Таким образом, системы цифрового туризма решают комплекс задач, охватывающий представление, визуализацию, маршрутизацию, персонализацию, автономность и аналитику туристической активности. Их внедрение даёт возможность перейти от фрагментированной и устаревшей информационной поддержки туризма к динамичной, масштабируемой и ориентированной на пользователя среде, способной повысить качество туристических услуг и уровень удовлетворённости посетителей.

\subsection{Потенциал для цифровизации туризма в Курской области}

В условиях растущей цифровизации всех сфер экономики особую актуальность приобретает переход туристической отрасли на современные технологические рельсы. Для Курской области цифровизация туризма — не просто тренд, а необходимое условие роста конкурентоспособности региона на внутреннем и внешнем туристическом рынке. Несмотря на существующие проблемы, потенциал для внедрения цифровых решений здесь достаточно высок.

Распространённость мобильных устройств и интернета.
По статистике, уровень проникновения интернета в регионе стабильно высок — более 85 \% домохозяйств имеют доступ к Сети, а в городах этот показатель приближается к 100 \%. Практически все туристы пользуются смартфонами с геолокацией и готовы взаимодействовать с информацией в цифровой форме. Это создаёт основу для:
\begin{itemize}
	\item мобильных приложений и веб-сервисов;
	\item интерактивных карт;
	\item цифровых путеводителей и аудиогидов;
	\item онлайн-бронирования и отзывов.
\end{itemize}

Объективная потребность в цифровых инструментах.
Туристы, посещающие регион, часто планируют поездку самостоятельно, без посредничества туроператоров. Это порождает спрос на удобные онлайн-инструменты, с помощью которых можно:
\begin{itemize}
	\item найти интересные места;
	\item построить маршрут;
	\item получить навигационные подсказки;
	\item ознакомиться с описанием, фото и расписанием объектов;
	\item сохранить избранные локации или поделиться ими.
\end{itemize}
Отсутствие таких сервисов\cite{b3} сегодня ощущается особенно остро, и их внедрение может кардинально изменить туристический опыт.

Наличие базовых цифровых активов.
Курская область уже располагает рядом ресурсов, которые могут быть интегрированы или дополнены в рамках цифровой платформы:
\begin{itemize}
	\item сайты администраций и учреждений культуры;
	\item открытые базы данных с геокоординатами и описаниями объектов;
	\item профили учреждений в соцсетях;
	\item фотографии и отзывы на сторонних платформах (Яндекс.Карты, Google Maps, Tripadvisor).
\end{itemize}
Это создаёт базу для формирования централизованной цифровой витрины достопримечательностей региона.

Интерес к цифровому туризму со стороны учреждений и властей.
В стратегических документах региона подчёркивается важность развития туризма и цифровой среды. Например, в рамках нацпроекта «Цифровая экономика» и программ поддержки туризма на 2021–2030 гг. отмечается необходимость создания интерактивных платформ и информационных сервисов. Это открывает возможности для грантов, программ поддержки стартапов и сотрудничества с государственными структурами.

Возможности локальной разработки.
В Курске и области действует ряд IT-компаний, а также вузов, готовящих специалистов по программной инженерии и цифровым технологиям. Это создаёт предпосылки для:
\begin{itemize}
	\item разработки региональных цифровых продуктов силами местных разработчиков;
	\item стажировок и студенческих проектов;
	\item более глубокой адаптации продуктов к локальным условиям.
\end{itemize}

Актуальность персонализированных и гибких решений.
Современные туристы ориентированы на самостоятельный, гибкий, персонализированный отдых. Цифровые технологии позволяют формировать индивидуальные маршруты, выбирать интересы (музеи, природа, религиозные объекты и т. д.), сохранять прогресс и получать актуальную информацию. Это особенно важно для региона, где объекты туризма распределены по территории и редко связаны в типовые маршруты.

\subsection{Сравнительный анализ аналогов}

Для более точного понимания требований к разрабатываемому веб-приложению с туристической картой Курской области необходимо изучить существующие цифровые решения в смежной области. Их функциональные возможности, преимущества и ограничения позволяют выработать чёткое представление о том, каким должен быть эффективный региональный сервис. Рассмотрим три наиболее близких по функциональности и назначению системы.

Яндекс.Путешествия.

«Яндекс.Путешествия» — это сервис от компании Яндекс, объединяющий функции планирования путешествий, бронирования и изучения интересных мест. Платформа ориентирована на российскую и международную аудиторию и предлагает пользователям информацию об отелях, маршрутах, достопримечательностях и туристических направлениях. На основе интеграции с другими сервисами Яндекса (в частности, Яндекс.Картами и Яндекс.Погодой) приложение формирует персонализированные подборки и советы.

Преимущества:
\begin{itemize}
	\item удобная и быстрая навигация благодаря знакомому интерфейсу Яндекс.Карт;
	\item автоматическая генерация маршрутов и подбор направлений;
	\item интеграция с погодными данными и сервисами бронирования;
	\item возможность просмотра фотографий и отзывов от других пользователей.
\end{itemize}

Недостатки:
\begin{itemize}
	\item основной упор сделан на коммерческие функции (отели, билеты), а не на независимый туризм;
	\item слабая проработка малых населённых пунктов и малоизвестных объектов;
	\item отсутствие функционала добавления собственных точек или построения кастомных маршрутов;
	\item невозможность сохранять свои маршруты и работать с избранным в полноценном виде.
\end{itemize}

Сервис «Яндекс.Путешествия» удобен для массового пользователя, но недостаточно гибок в плане самостоятельного планирования уникальных маршрутов, особенно на уровне одного региона. Кроме того, он не предоставляет возможности для активного участия пользователя в формировании базы точек, что ограничивает глубину взаимодействия с территорией.

Izi.TRAVEL.

Международная платформа, ориентированная на аудиогиды и цифровые маршруты. Приложение позволяет прослушивать экскурсии, следовать по заранее заданным трекам, использовать GPS-навигацию и просматривать описания объектов. Имеет веб-версию и мобильное приложение.

Преимущества:
\begin{itemize}
	\item возможность прослушивания информации в аудиоформате, в том числе оффлайн;
	\item широкое географическое покрытие, включая малые города и музейные маршруты;
	\item простое и понятное управление маршрутами.
\end{itemize}

Недостатки:
\begin{itemize}
	\item сложности с добавлением пользовательского контента;
	\item интерфейс устаревшего дизайна, не всегда удобный при активном взаимодействии;
	\item ограниченная возможность создания кастомных маршрутов (только по заранее заданным точкам).
\end{itemize}

Izi.TRAVEL — полезный инструмент для аудиоэкскурсий, но его узкая специализация и статичность маршрутов ограничивают функциональность в контексте самостоятельного туризма.

Tripster.ru.

Коммерческий сервис, специализирующийся на индивидуальных экскурсиях и авторских маршрутах. Он предлагает пользователю выбрать гида, маршрут, формат экскурсии, а также просматривать отзывы и бронировать мероприятия напрямую на платформе.

Преимущества:
\begin{itemize}
	\item широкий выбор индивидуальных и групповых экскурсий;
	\item развитая система фильтрации и поиска по интересам;
	\item возможность общения с гидами и чтения отзывов.
\end{itemize}

Недостатки:
\begin{itemize}
	\item основной акцент — на платных услугах, а не на самостоятельном планировании;
	\item недоступность внутренней карты с независимыми точками и маршрутами;
	\item отсутствие возможности свободного добавления объектов и построения маршрутов пользователями.
\end{itemize}

Tripster эффективно решает задачи коммерческого экскурсионного сервиса, но его структура и бизнес-модель не ориентированы на самостоятельного путешественника, нуждающегося в универсальном навигационном инструменте.

\subsection{Обоснование необходимости собственного решения}

Несмотря на большое количество существующих цифровых платформ, так или иначе связанных с туристической навигацией, большинство из них разрабатываются с ориентацией на массового пользователя, масштабные маршруты и коммерческую составляющую. Универсальные сервисы — такие как Яндекс.Путешествия, Tripster или Izi.TRAVEL — действительно охватывают широкие географические территории, предоставляют доступ к популярным направлениям и справочной информации. Однако, при более детальном рассмотрении становится очевидно, что все они имеют существенные ограничения, если рассматривать их применение в контексте локального туризма, особенно на уровне небольших городов и сельских территорий.

Во-первых, глобальные сервисы в большинстве случаев не обладают полной и актуальной информацией о малоизвестных, но значимых объектах региона, которые могут представлять интерес как для жителей, так и для гостей. Часто такие точки попросту отсутствуют на карте, не имеют фотографий, описания или маршрутов подъезда. Во-вторых, доступные пользователю функции в большинстве аналогов ограничены просмотром, без возможности самостоятельно влиять на структуру данных — добавлять новые объекты, сохранять маршрутные схемы, фильтровать объекты по категориям в зависимости от интересов.

Также стоит отметить отсутствие у существующих сервисов полноценной функции пользовательского построения маршрутов с промежуточными точками, что особенно важно в условиях региональных поездок, когда путешествие строится не по магистральным маршрутам, а через локальные достопримечательности. Такие сценарии невозможно реализовать в типичных предложениях крупных платформ без значительных ограничений. Кроме того, функции работы с избранным, отзывами, визуальными подсказками часто либо отсутствуют, либо представлены в минималистичной форме.

Особую значимость имеет тот факт, что ни один из существующих сервисов не адаптирован специально под условия конкретного региона, его туристическую инфраструктуру, локальную специфику, язык и визуальный стиль. В результате пользователь, планирующий поездку, сталкивается либо с избытком нерелевантной информации, либо с её полным отсутствием. Также ограничена обратная связь с местным сообществом, учреждениями культуры или туристическими центрами, что важно в условиях цифрового взаимодействия и территориального маркетинга.

Создание собственного веб-приложения позволяет устранить все перечисленные недостатки. Оно предоставляет возможность сформировать полноценную цифровую карту региона, включающую реальные, локальные объекты интереса, привязанные к географическим координатам, фотографиям, категориям и пользовательским данным. Веб-интерфейс приложения создаётся с нуля — с учётом особенностей региональной аудитории, возможностей мобильного доступа, потребности в простом и понятном управлении маршрутом.

Кроме того, самостоятельная разработка открывает возможности для интеграции с региональными ресурсами, в том числе учреждениями культуры, местными администрациями, музейными системами. Это создаёт задел для расширения и масштабирования проекта в рамках региональных программ цифровизации и туристического продвижения.

Таким образом, разработка собственного решения — это не просто техническая альтернатива существующим продуктам, а целенаправленный ответ на локальные информационные потребности, направленный на популяризацию туристического потенциала, повышение уровня осведомлённости пользователей и развитие цифровой инфраструктуры в регионе.


