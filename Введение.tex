\section*{ВВЕДЕНИЕ}
\addcontentsline{toc}{section}{ВВЕДЕНИЕ}

Развитие информационных технологий оказывает всё более ощутимое влияние на различные сферы жизни общества, включая туризм. В условиях цифровой трансформации особую актуальность приобретает задача создания современных, удобных и доступных средств цифровой навигации, позволяющих пользователю не только получать информацию об объектах, но и выстраивать индивидуальные маршруты, сохранять интересные места и взаимодействовать с территорией в интерактивной форме.

На сегодняшний день путешественники и туристы всё чаще прибегают к помощи онлайн-карт, мобильных приложений и веб-сервисов для планирования поездок, знакомства с достопримечательностями и оценки культурных маршрутов. Это особенно актуально в контексте внутреннего туризма, где цифровые решения становятся инструментом популяризации малоизвестных, но ценных исторических, природных и архитектурных объектов.

Однако большинство существующих платформ ориентированы либо на массовые направления и коммерческий контент, либо на крупные города, оставляя за пределами внимания локальные регионы с их уникальным потенциалом. В результате пользователи нередко сталкиваются с недостатком актуальной информации, невозможностью настроить маршрут под свои интересы или отсутствием простого и удобного интерфейса.

В условиях перехода к более гибким и персонифицированным моделям взаимодействия с туристом возникает потребность в специализированных цифровых продуктах, создаваемых с учётом локальной специфики, особенностей региона и потребностей реальных пользователей. Такие решения должны быть легковесными, доступными, интуитивно понятными и в то же время функционально насыщенными — предоставлять возможности фильтрации объектов, построения маршрутов с промежуточными точками, просмотра информации и взаимодействия с картой в реальном времени.

Целью данной работы является разработка веб-приложения, позволяющего пользователям взаимодействовать с туристической картой Курской области, просматривать достопримечательности, фильтровать их по категориям, строить маршруты различного типа и сохранять избранные места. Проект нацелен на создание простой в использовании и в то же время гибкой системы, способной повысить доступность туристической информации и стимулировать интерес к внутреннему туризму.

В процессе выполнения работы необходимо решить следующие задачи:
\begin{itemize}
\item исследовать предметную область и существующие цифровые решения в сфере туризма;
\item сформировать функциональные и технические требования к программной системе;
\item разработать архитектуру клиентской и серверной частей веб-приложения;
\item реализовать интерфейс взаимодействия с картой и маршрутным функционалом;
\item разработать программные модули клиентской и серверной частей веб-приложения;
\item провести тестирование и интеграцию компонентов в единую систему.
\end{itemize}

Практическая значимость проекта заключается в его универсальности, адаптируемости под другие регионы и возможности дальнейшего расширения функционала — от внедрения пользовательской авторизации до подключения аналитики и оффлайн-режимов. Разработка такой системы может быть использована как основа для региональных туристических платформ и образовательных проектов, направленных на цифровизацию территории.

Таким образом, актуальность темы обусловлена одновременно технологическими, социальными и культурными факторами, а выполненная разработка представляет собой не только программное решение, но и вклад в развитие цифровой инфраструктуры туризма на уровне региона.

\emph{Структура и объем работы.} Отчет состоит из введения, 4 разделов основной части, заключения, списка использованных источников, 2 приложений. Текст выпускной квалификационной работы равен \formbytotal{lastpage}{страниц}{е}{ам}{ам}.

\emph{Во введении} сформулирована цель работы, поставлены задачи разработки, описана структура работы, приведено краткое содержание каждого из разделов.

\emph{В первом разделе} проводится анализ предметной области. Здесь раскрывается актуальность цифровизации в сфере туризма, определяются основные задачи информационных систем, ориентированных на туристов, а также приводится сравнительный анализ существующих аналогов. Особое внимание уделяется выявлению недостатков популярных решений и обоснованию необходимости создания собственного приложения, адаптированного под локальный уровень — на примере Курской области.

\emph{Во втором разделе} формируется техническое задание на разработку системы. Описываются цель и назначение проекта, формулируются функциональные и нефункциональные требования, приводятся сценарии использования, диаграмма прецедентов и описание пользовательских ролей. Также рассматриваются ключевые элементы взаимодействия пользователя с системой и логика её функционирования.

\emph{В третьем разделе} рассматривается технический проект программной системы. Приводится обоснование выбора технологического стека, описывается архитектура веб-приложения, компоненты и их взаимодействие. Выполнено проектирование пользовательского интерфейса, структуры данных и маршрутов. Отдельный акцент сделан на интеграции с API Яндекс.Карт, а также на построении маршрутов и работе с интерактивной картой. Раздел сопровождается диаграммами компонентов, развертывания и концептуальной моделью данных.

\emph{В четвертом разделе} подводятся итоги разработки. Рассматривается процесс тестирования системы, демонстрируются основные реализованные функции, оценивается соответствие полученного результата поставленным задачам. Также предлагаются направления возможной доработки и развития проекта в рамках расширения функциональности или адаптации под другие регионы.

В заключении излагаются основные результаты работы, полученные в ходе разработки.

В приложении А представлен графический материал.
В приложении Б представлены фрагменты исходного кода. 
