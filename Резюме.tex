\section*{РЕЗЮМЕ}
\addcontentsline{toc}{section}{РЕЗЮМЕ}

Название: «Бизнес-проект «Внедрение цифровых технологий в сфере туризма» представляет собой веб-приложение, предназначенное для визуализации туристических объектов Курской области и построения персонализированных маршрутов. В основу разработки положены современные технологии веб-картографирования и пространственного анализа данных. Система предоставляет пользователям — жителям региона, туристам, органам власти и туристическим организациям — интерактивный инструмент для поиска, навигации и анализа туристических ресурсов.

Особое внимание в проекте уделено исследованию механизма государственного регулирования сферы туризма в условиях цифровизации. Разработаны рекомендации по совершенствованию цифровой инфраструктуры государственного регулирования, направленные на повышение доступности информации о туристическом потенциале региона и поддержку устойчивого развития туризма.

Актуальность: несмотря на высокий туристический потенциал Курской области, регион остаётся недостаточно представленным в цифровом пространстве. Отсутствие единого ресурса затрудняет доступ туристов к актуальной информации, а органы власти не имеют эффективных инструментов для мониторинга и продвижения туристических маршрутов. Разработка специализированного веб-приложения позволит устранить эти барьеры, повысить качество государственного регулирования в сфере туризма и усилить конкурентоспособность региона.

Цель: создать удобный инструмент для планирования путешествий по Курской области, обеспечивающий доступ к информации о малоизвестных культурно-исторических, природных и локальных туристических объектах с помощью персонализированных маршрутов. Оно направлено на популяризацию внутреннего туризма, развитие региональной экономики через интеграцию с малым бизнесом и формирование устойчивой экосистемы, сочетающей технологические инновации с сохранением культурного наследия и привлечением как российских, так и международных путешественников.

В ходе работы были решены следующие задачи:
\begin{itemize}
	\item проведён анализ туристических объектов Курской области и их пространственных характеристик;
	\item исследованы методы хранения и обработки пространственных и иных данных;
	\item выполнено проектирование архитектуры системы и пользовательского интерфейса;
	\item реализовано веб-приложение с использованием Яндекс.Карт API и СУБД MySQL;
	\item проведено тестирование производительности системы;
	\item проведено исследование механизмов цифровизации государственного регулирования сферы туризма и разработаны предложения по их совершенствованию.
\end{itemize}

Стратегия продвижения стартап проекта базируется на интеграции технологических, маркетинговых и социокультурных подходов. Ключевыми элементами стратегии являются партнёрские сети, охватывающие государственные структуры, такие как Министерство культуры и Министерство приоритетных проектов развития территорий и туризма Курской области, для обеспечения актуальности данных о мероприятиях и объектах наследия. Сотрудничество с образовательными учреждениями направлено на разработку тематических маршрутов для школьных и студенческих групп, формируя лояльность среди молодёжи.

Уникальность продукта: веб-приложение «Россия твоими глазами» представляет собой интерактивную карту предоставляющую пользователям возможность находить интересные туристические объекты по категории, названию или местоположению, просматривать подробную информацию о достопримечательностях, создавать и сохранять персонализированные маршруты, использовать геолокацию для построения маршрутов от текущего местоположения пользователя. В отличие от глобальных аналогов, приложение фокусируется на уникальной региональной специфике Курской области, включая малоизвестные достопримечательности, что обеспечивает его конкурентоспособность в нише внутреннего туризма.

Предполагаемые результаты стартап-проекта: реализация проекта направлена на повышение туристического потока за счёт популяризации скрытых локаций и оптимизации информационного доступа. Приложение стимулирует развитие локальной экономики через партнёрство с малым бизнесом (рестораны, сувенирные лавки, транспортные сервисы) и активизацию событийного туризма. Формируется устойчивая экосистема внутреннего туризма, сочетающая технологические инновации с сохранением культурно-исторического наследия. Социальный эффект проявляется в усилении региональной идентичности среди жителей и привлечении молодёжи к изучению родного края через интерактивные форматы. 

Горизонт расчета результатов стартап-проекта. Краткосрочные результаты (1–2 года) связаны с массовым проникновением приложения среди целевой аудитории, формированием базы данных о туристических объектах и установлением партнёрских связей с государственными и частными структурами. 

Среднесрочные цели (3–5 лет) включают масштабирование функционала на федеральный уровень, монетизацию через подписки и рекламные интеграции, а также устойчивый рост туристического потока в Курской области. 

Долгосрочные эффекты (5+ лет) предполагают трансформацию приложения в национальную платформу для внутреннего туризма с расширенной функциональностью, включая AR/VR-визуализацию маршрутов, укрепление позиций регионов РФ как центра культурно-познавательного и экологического туризма, снижение сезонной зависимости туристического потока. 

Источники финансирования стартап-проекта веб-приложения «Россия твоими глазами» предполагают многоуровневую модель, сочетающую государственную поддержку, частные инвестиции и рыночные механизмы монетизации. Основным источником выступают целевые субсидии из регионального бюджета Курской области, направляемые на развитие туристической инфраструктуры. Дополнительно планируется привлечение внебюджетных средств через партнёрские программы с местными предприятиями гостеприимства, культурными учреждениями и транспортными операторами, заинтересованными в интеграции своих сервисов в экосистему приложения. Финансирование также может осуществляться за счёт грантов федеральных программ, ориентированных на популяризацию внутреннего туризма и внедрение инновационных технологий в региональное развитие.

Условия финансирования стартап-проекта предполагают выполнение строгих критериев эффективности, включая регулярное обновление пространственных данных, обеспечение пользовательского охвата и достижение показателей вовлечённости целевой аудитории. Государственные средства выделяются на конкурсной основе с обязательным предоставлением отчётов о реализации этапов проекта, включая техническую поддержку и расширение базы туристических объектов. Частные инвестиции предполагают создание устойчивой модели доходности через подписку на премиум-функции, размещение таргетированной рекламы локальных брендов.

Наличие интеллектуальной собственности. Интеллектуальная собственность приложения не оформлена явно, но имеет потенциал для защиты при переходе к практической реализации и масштабированию.

Интегральные показатели экономической эффективности стартап-проекта. Расчеты экономической эффективности мобильного приложения «Россия твоими глазами» подтверждают его финансовую целесообразность:
\begin{itemize}
	\item чистый дисконтированный доход (NPV) составляет 2,5 млн рублей при ставке дисконтирования 10\%;
	\item внутренняя норма доходности (IRR) – 28\%, что превышает среднерыночные показатели;
	\item срок окупаемости – 2,5 года;
	\item индекс рентабельности (PI) – 1,5;
	\item совокупный возврат инвестиций (ROI) — 100\% за пять лет.  
\end{itemize}

Косвенные эффекты включают рост туристического потока на 20–30\%, увеличение выручки малого бизнеса на 10–15\%, создание рабочих мест в IT и смежных сферах. Региональный бренд усиливается через цифровую платформу.

Риски проведения стартап-проекта. Проект сталкивается с рисками, связанными с технологической зависимостью, кибербезопасностью и конкуренцией. Сбои в работе серверов, геолокационных сервисов или алгоритмов маршрутизации могут снизить доверие пользователей, особенно на фоне устоявшихся аналогов. Критичной уязвимостью является необходимость регулярного обновления данных о туристических объектах и инфраструктуре – их устаревание снизит полезность приложения, особенно в динамичных регионах. Угрозы кибербезопасности, такие как утечка персональных данных, требуют внедрения надежных протоколов защиты. Быстрое развитие технологий (AR/VR, ИИ) создает риск морального устаревания продукта без своевременной интеграции инноваций. Низкая узнаваемость бренда на старте может затруднить привлечение аудитории в условиях доминирования глобальных платформ.

Потенциал проекта. Проект обладает потенциалом трансформировать туристическую экосистему Курской области и масштабироваться на федеральный уровень. Фокус на локальную специфику позволяет раскрыть уникальные культурно-исторические и природные объекты через персонализированные маршруты и интерактивный контент, стимулируя внутренний туризм и экономическое развитие региона. 

Экономический потенциал реализуется через монетизацию подписок, премиум-функций и рекламы, обеспечивающих финансовую устойчивость. Сотрудничество с государственными структурами, образовательными учреждениями и местным бизнесом расширяет функциональность, интегрируя приложение в системы образования, наследия и инфраструктуры. В долгосрочной перспективе проект может стать национальной платформой, объединяющей региональные инициативы и способствуя развитию цифрового туризма в России.  


