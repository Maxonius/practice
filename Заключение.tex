\section*{ЗАКЛЮЧЕНИЕ}
\addcontentsline{toc}{section}{ЗАКЛЮЧЕНИЕ}

В процессе выполнения выпускной квалификационной работы было спроектировано и реализовано веб-приложение, ориентированное на цифровую поддержку туристической активности в пределах региона. Результатом проекта стала информационная система, предоставляющая пользователю интерактивную карту с отображением достопримечательностей, функцией построения маршрутов и возможностью управления избранными объектами.

Работа включала в себя всесторонний анализ предметной области, исследование актуальных цифровых решений в сфере туризма и выявление требований к создаваемому программному продукту. На основании полученных данных было сформировано техническое задание и обоснована архитектура разрабатываемой системы.

В ходе работы были получены следующие результаты:

\begin{enumerate}
\item Проведен анализ предметной области и существующих цифровых решений в сфере туризма.
\item Сформированы функциональные и технические требования к программной системе.
\item Разработана архитектура клиентской и серверной частей веб-приложения.
\item Реализован интерфейс взаимодействия с картой и маршрутным функционалом.
\item Разработаны программные модули клиентской и серверной частей веб-приложения.
\item Проведены тестирование и интеграция компонентов в единую систему.
\end{enumerate}

Таким образом, разработанное веб-приложение демонстрирует применение современных веб-технологий в сфере цифрового туризма. Решение обладает высокой степенью прикладной значимости и может служить основой для дальнейшего развития информационных сервисов, ориентированных на поддержку и продвижение локального туризма. Благодаря гибкой архитектуре и модульному построению, система может быть легко адаптирована под географические особенности других регионов, дополнена новыми пользовательскими сценариями и функциональными модулями. Кроме того, потенциал проекта позволяет реализовать мобильную версию приложения, взаимодействие со сторонними платформами, а также внедрение аналитических инструментов для мониторинга туристической активности и предпочтений пользователей.