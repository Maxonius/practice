\section*{ЗАКЛЮЧЕНИЕ}
\addcontentsline{toc}{section}{ЗАКЛЮЧЕНИЕ}

В процессе выполнения выпускной квалификационной работы было спроектировано и реализовано веб-приложение, ориентированное на цифровую поддержку туристической активности в пределах региона. Результатом проекта стала информационная система, предоставляющая пользователю интерактивную карту с отображением достопримечательностей, функцией построения маршрутов и возможностью управления избранными объектами.

Работа включала в себя всесторонний анализ предметной области, исследование актуальных цифровых решений в сфере туризма и выявление требований к создаваемому программному продукту. На основании полученных данных было сформировано техническое задание и обоснована архитектура разрабатываемой системы.

В ходе работы были выполнены следующие основные задачи:

\begin{enumerate}
\item Проведён анализ туристической инфраструктуры региона и обоснована необходимость цифровизации.
\item Сформированы функциональные и технические требования к системе, включая пользовательские сценарии.
\item Разработана архитектура веб-приложения с учётом взаимодействия клиентской части, серверной логики и внешнего API.
\item Реализован пользовательский интерфейс с возможностью фильтрации объектов, отображения балунов, построения и сохранения маршрутов.
\item Проведено тестирование модулей и всей системы в целом, а также выполнена сборка программных компонентов в единый продукт.
\end{enumerate}

Разработанное приложение успешно демонстрирует применение современных веб-технологий в сфере цифрового туризма и обладает потенциалом дальнейшего развития. Система может быть адаптирована под другие регионы, расширена новыми функциями, а также дополнена мобильным интерфейсом и поддержкой сторонних сервисов.
