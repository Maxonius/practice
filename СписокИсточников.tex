\addcontentsline{toc}{section}{СПИСОК ИСПОЛЬЗОВАННЫХ ИСТОЧНИКОВ}

\begin{thebibliography}{21}

    \bibitem{b1} Всемирная туристическая организация (UNWTO). Доклад «Tourism and Technology Outlook 2023» : сайт. – URL: https://www.unwto.org/tourism-and-technology-outlook-2023 (дата обращения: 09.03.2025). – Текст : электронный.
   	\bibitem{b2}	UNWTO. Tourism and Sustainable Development Goals – Progress, Challenges and Opportunities : сайт. – URL: https://www.unwto.org/tourism-and-sdgs-report (дата обращения: 09.06.2025). – Текст : электронный.
   	\bibitem{b3}	Смит, А. Современная JavaScript-разработка: от ES6 до ES2022 / А. Смит. – СПб. : Питер, 2023. – 384 с. – ISBN 978-5-4461-1971-7. – Текст : непосредственный.
    \bibitem{b4}	Сандерс, Р. Node.js. Создание серверных приложений на JavaScript / Р. Сандерс. – СПб. : Питер, 2021. – 320 с. – ISBN 978-5-4461-1605-1. – Текст : непосредственный.
   	\bibitem{b5}	Mozilla Developer Network : Web APIs : сайт. – URL: https://developer.mozilla.org/en-US/docs/Web/API (дата обращения: 01.05.2025). – Текст : электронный.
	\bibitem{b6}	Джонсон, Д. Веб-дизайн: современные подходы к созданию интерфейсов / Д. Джонсон. – М. : Эксмо, 2020. – 352 с. – ISBN 978-5-04-108539-9. – Текст : непосредственный.
	\bibitem{b7} Норман, Д. Дизайн привычных вещей / Д. Норман. – М. : Манн, Иванов и Фербер, 2016. – 384 с. – ISBN 978-5-00057-684-7. – Текст : непосредственный.
   	\bibitem{b8}	Рейнджер, М. Проектирование пользовательского опыта: Руководство по UX / М. Рейнджер. – М. : Альпина Паблишер, 2021. – 278 с. – ISBN 978-5-9614-7183-3. – Текст : непосредственный.    
	\bibitem{b9} Бозетти, Ф. UX Design 2022: Руководство по проектированию пользовательского интерфейса / Ф. Бозетти. – Independently Published, 2022. – 206 с. – ISBN 979-8-3912-8453-0. – Текст : непосредственный.
	\bibitem{b10}	Коупленд, Л. Архитектура REST API. Проектирование надёжных веб-сервисов / Л. Коупленд. – М. : Диалектика, 2018. – 288 с. – ISBN 978-5-8459-2006-4. – Текст : непосредственный.
 	\bibitem{b11} Яндекс.Карты API : JavaScript API 2.1 : сайт. – URL: https://yandex.ru/dev/maps/jsapi/doc/2.1/quick-start/ (дата обращения: 27.04.2025). – Текст : электронный.
 	\bibitem{b12}	Вуд, А. CSS: Секреты профессионалов. 2-е изд. / А. Вуд. – М. : Питер, 2019. – 400 с. – ISBN 978-5-4461-1383-8. – Текст : непосредственный.
 	\bibitem{b13} Сноу, М. Архитектура клиент-серверных приложений: современные подходы и технологии / М. Сноу. – СПб. : Диалектика, 2021. – 312 с. – ISBN 978-5-6046702-3-4. – Текст : непосредственный.	
 	\bibitem{b19} HTML Living Standard : WHATWG specification : сайт. – URL: https://html.spec.whatwg.org/ (дата обращения: 21.04.2025). – Текст : электронный.
 	\bibitem{b18} Фланаган, Д. JavaScript. Подробное руководство. 7-е изд. / Д. Фланаган. – М. : Вильямс, 2021. – 960 с. – ISBN 978-5-8459-2231-0. – Текст : непосредственный.
	\bibitem{b14}	Хартманн, К. Базы данных для веб-разработчиков: от SQL до NoSQL / К. Хартманн. – СПб. : БХВ-Петербург, 2020. – 352 с. – ISBN 978-5-9775-4055-3. – Текст : непосредственный. 
	\bibitem{b15} Нильсен, Я. Инженерия удобства использования / Я. Нильсен. – М. : Лори, 2017. – 368 с. – ISBN 978-5-8114-2475-5. – Текст : непосредственный.
	\bibitem{b16} MDN Web Docs: Cascading Style Sheets (CSS) : сайт. – URL: https://developer.mozilla.org/en-US/docs/Web/CSS (дата обращения: 02.05.2025). – Текст : электронный.
	\bibitem{b17} JavaScript: The Definitive Guide / Д. Флэнаган. – 7-е изд. – O’Reilly Media, 2020. – 706 с. – ISBN 978-1-4919-5542-8. – Текст : непосредственный.
	
	\bibitem{b20} Купер, А. Интерфейс: основы проектирования взаимодействия / А. Купер, Р. Райман, Д. Кронин. – 4-е изд. – СПб. : Питер, 2019. – 720 с. – ISBN 978-5-4461-0927-6. – Текст : непосредственный.
	
	\bibitem{b21} Мердок, С. JavaScript: карманный справочник / С. Мердок. – 6-е изд. – М. : Символ-Плюс, 2021. – 288 с. – ISBN 978-5-93286-406-2. – Текст : непосредственный.
	\bibitem{b22} Харт, Дж. Современное руководство по CSS: адаптивная вёрстка и дизайн / Дж. Харт. – М. : ДМК Пресс, 2020. – 376 с. – ISBN 978-5-97060-870-3. – Текст : непосредственный.
	\bibitem{b23} Дакетт, Дж. HTML и CSS. Разработка и дизайн веб-сайтов / Дж. Дакетт. – М. : Эксмо, 2018. – 512 с. – ISBN 978-5-699-93248-3. – Текст : непосредственный.
	\bibitem{b24} Танненбаум, Э. Архитектура современных операционных систем / Э. Танненбаум, Х. Бос. – СПб. : Питер, 2022. – 864 с. – ISBN 978-5-4461-2206-0. – Текст : непосредственный.
	\bibitem{b25} Стойков, П. Создание web-приложений на PHP 8 и MySQL / П. Стойков. – М. : Наука и техника, 2022. – 416 с. – ISBN 978-5-907363-14-7. – Текст : непосредственный.
	\bibitem{b26} Войлоков, С.Ю. Программирование на JavaScript. Практическое руководство / С.Ю. Войлоков. – М. : Наука и техника, 2021. – 352 с. – ISBN 978-5-94387-914-5. – Текст : непосредственный.
	\bibitem{b27} Кузнецов, Д. HTML и CSS: современный учебник / Д. Кузнецов. – СПб. : БХВ-Петербург, 2022. – 304 с. – ISBN 978-5-9775-6014-1. – Текст : непосредственный.
	\bibitem{b28} Брукс, С. Проектирование пользовательских интерфейсов: от идей к решениям / С. Брукс. – М. : Диалектика, 2019. – 280 с. – ISBN 978-5-8459-2089-7. – Текст : непосредственный.
	\bibitem{b29} Асташкин, П.В. Веб-программирование на PHP 8 и MySQL. Полное руководство / П.В. Асташкин. – М. : ДМК Пресс, 2021. – 416 с. – ISBN 978-5-97060-931-1. – Текст : непосредственный.
	\bibitem{b30} Малинин, Н.В. Геоинформационные системы и картографические сервисы / Н.В. Малинин. – М. : Горячая линия – Телеком, 2020. – 280 с. – ISBN 978-5-9912-1014-7. – Текст : непосредственный.

\end{thebibliography}
