\addcontentsline{toc}{section}{СПИСОК ИСПОЛЬЗОВАННЫХ ИСТОЧНИКОВ}

\begin{thebibliography}{21}

    \bibitem{b1} Фаулер, М. Архитектура корпоративных программных приложений / М. Фаулер. – 3-е изд., перераб. – СПб. : Питер, 2020. – 544 с. – ISBN 978-5-4461-1692-1. – Текст : непосредственный.
   	\bibitem{b2}	Ларсен, У. Архитектура веб-приложений: от монолита к микросервисам / У. Ларсен. – М. : БХВ-Петербург, 2019. – 416 с. – ISBN 978-5-9775-4009-6. – Текст : непосредственный.
   	\bibitem{b3}	Смит, А. Современная JavaScript-разработка: от ES6 до ES2022 / А. Смит. – СПб. : Питер, 2023. – 384 с. – ISBN 978-5-4461-1971-7. – Текст : непосредственный.
    \bibitem{b4}	Сандерс, Р. Node.js. Создание серверных приложений на JavaScript / Р. Сандерс. – СПб. : Питер, 2021. – 320 с. – ISBN 978-5-4461-1605-1. – Текст : непосредственный.
   	\bibitem{b5}	Mozilla Developer Network : Web APIs : сайт. – URL: https://developer.mozilla.org/en-US/docs/Web/API (дата обращения: 01.05.2025). – Текст : электронный.
	\bibitem{b6}	Джонсон, Д. Веб-дизайн: современные подходы к созданию интерфейсов / Д. Джонсон. – М. : Эксмо, 2020. – 352 с. – ISBN 978-5-04-108539-9. – Текст : непосредственный.
	\bibitem{b7} Норман, Д. Дизайн привычных вещей / Д. Норман. – М. : Манн, Иванов и Фербер, 2016. – 384 с. – ISBN 978-5-00057-684-7. – Текст : непосредственный.
   	\bibitem{b8}	Рейнджер, М. Проектирование пользовательского опыта: Руководство по UX / М. Рейнджер. – М. : Альпина Паблишер, 2021. – 278 с. – ISBN 978-5-9614-7183-3. – Текст : непосредственный.    
	\bibitem{b9} Бозетти, Ф. UX Design 2022: Руководство по проектированию пользовательского интерфейса / Ф. Бозетти. – Independently Published, 2022. – 206 с. – ISBN 979-8-3912-8453-0. – Текст : непосредственный.
	\bibitem{b10}	Коупленд, Л. Архитектура REST API. Проектирование надёжных веб-сервисов / Л. Коупленд. – М. : Диалектика, 2018. – 288 с. – ISBN 978-5-8459-2006-4. – Текст : непосредственный.
 	\bibitem{b11} Яндекс.Карты API : JavaScript API 2.1 : сайт. – URL: https://yandex.ru/dev/maps/jsapi/doc/2.1/quick-start/ (дата обращения: 27.04.2025). – Текст : электронный.
 	\bibitem{b12}	Вуд, А. CSS: Секреты профессионалов. 2-е изд. / А. Вуд. – М. : Питер, 2019. – 400 с. – ISBN 978-5-4461-1383-8. – Текст : непосредственный.
 	\bibitem{b13} Сноу, М. Архитектура клиент-серверных приложений: современные подходы и технологии / М. Сноу. – СПб. : Диалектика, 2021. – 312 с. – ISBN 978-5-6046702-3-4. – Текст : непосредственный.	
 	\bibitem{b19} HTML Living Standard : WHATWG specification : сайт. – URL: https://html.spec.whatwg.org/ (дата обращения: 21.04.2025). – Текст : электронный.
 	\bibitem{b18} Фланаган, Д. JavaScript. Подробное руководство. 7-е изд. / Д. Фланаган. – М. : Вильямс, 2021. – 960 с. – ISBN 978-5-8459-2231-0. – Текст : непосредственный.
	\bibitem{b14}	Хартманн, К. Базы данных для веб-разработчиков: от SQL до NoSQL / К. Хартманн. – СПб. : БХВ-Петербург, 2020. – 352 с. – ISBN 978-5-9775-4055-3. – Текст : непосредственный. 
	\bibitem{b15} Nielsen, J. Usability Engineering / J. Nielsen. – Morgan Kaufmann, 2016. – 362 p. – ISBN 978-0125184069. – Текст : непосредственный.
	\bibitem{b16} MDN Web Docs: Cascading Style Sheets (CSS) : сайт. – URL: https://developer.mozilla.org/en-US/docs/Web/CSS (дата обращения: 02.05.2025). – Текст : электронный.
	\bibitem{b17} JavaScript: The Definitive Guide / Д. Флэнаган. – 7-е изд. – O’Reilly Media, 2020. – 706 с. – ISBN 978-1-4919-5542-8. – Текст : непосредственный.
	
	\bibitem{b20} Cooper, A. About Face: The Essentials of Interaction Design / A. Cooper, R. Reimann, D. Cronin. – 4th ed. – Wiley, 2017. – 720 p. – ISBN 978-1118766576. – Текст : непосредственный.

\end{thebibliography}
